\documentclass[10pt, letterpaper]{article}

\RequirePackage{fix-cm}
\usepackage[fontsize=10.5pt]{scrextend}
%
% Preamble
%
\usepackage{graphicx}

\usepackage[hmargin=1cm, vmargin=1.5cm]{geometry}
\usepackage[hidelinks]{hyperref}
\usepackage[absolute]{textpos} %TPHorizontal....
\usepackage{anyfontsize} % to control fonts
\usepackage{setspace} % to control the spacing between lines
\usepackage{pbox}

\newlength{\spacebox} % define a new length
\settowidth{\spacebox}{8888888888}          % Box to align text
\newcommand{\PersonalEntry}[2]{
    \noindent\hangindent=10cm\hangafter=0 % Indentation
    \makebox[5em][l]{\textit{#1}}
    \hspace{1em}
    \mbox{#2} \par
}

\newcommand{\EducationEntry}[4]{
    \noindent\mbox{#1} \hfill
    \mbox{#2} \\
    \indent
    \mbox{\textit{#3}} \vspace{5pt}\\
    \parbox[t][\height][c]{\textwidth}{\small{#4}}
	\vspace{-14pt}
}

\newcommand{\ExperienceEntry}[5]{
    \noindent\mbox{#1} \hfill
    \mbox{#2} \\
    \if\relax\detokenize{#3}\relax\else
    \mbox{#3}\\
    \fi
    \indent
    \mbox{#4} \vspace{5pt}\\
    \parbox[t][\height][c]{\textwidth}{\small{#5}}
	\vspace{-1pt}
}

\newcommand{\Section}[1]{
 %   \vspace{5pt}
 %   \noindent\hangindent=0.5cm\hangafter=0
 %   \textbf{\fontsize{20}{20}\selectfont #1} \par
 %   \vspace{-5pt}
 %   \noindent\rule{\textwidth}{0.4pt} \par
 %   \vspace{5pt}
    \Line \par
    \vspace{1pt}
    \textbf{\fontsize{13}{13}\selectfont #1} \par
    \vspace{-7pt}
    \Line \par
    \vspace{3pt}
}

\newcommand{\Line}{
\noindent\rule{\textwidth}{0.4pt}}

\setlength{\TPHorizModule}{1mm}
\setlength{\TPVertModule}{1mm}
\textblockorigin{0mm}{5mm}
\newcommand{\lastupdated}{
    \begin{textblock}{100}(130,0)
    \fontsize{6pt}{8pt}
    Last updated on \today
    \end{textblock}
}

\newenvironment{tightitemize}
{
    \vspace{-\topsep}
    \begin{itemize}
        \itemsep2pt \parskip0pt \parsep0pt
}
{
    \end{itemize}
    \vspace{-\topsep}
}

%
% Document Environment
%
\begin{document}

\lastupdated

%
% Name
%
\textbf{\fontsize{25}{25}\selectfont Pratyush Das}
\vspace{0.7cm}

%
% Personal Entries
%
\vspace{-45pt} \begin{spacing}{1}
\PersonalEntry{\textbullet{} Email}{\href{mailto:das160@purdue.edu}{das160@purdue.edu}}
\PersonalEntry{\textbullet{} GitHub}{\urlstyle{same}\url{https://github.com/reikdas}}
\end{spacing}
\vspace{-13pt}

%
% Education
%
\vspace{10pt}
\Section{Education}
\vspace{-4pt}
\EducationEntry{\textbf{Purdue University}}{August, 2021 -}{PhD in Computer Science (Advisor - Milind Kulkarni)}{
    \vspace{-10pt}
    \hspace{30pt}Research interests - Compilers, Automatic parallelization, Sparse tensors, High performance computing
  }\\
\EducationEntry{\textbf{Institute of Engineering \& Management, Kolkata (MAKAUT)}}{August, 2017 - May, 2021}{Bachelor of Technology in Computer Science and Engineering}{
    \vspace{-10pt}
    \hspace{30pt}Awarded the Director's Award for Best Scientific Mind
}
\vspace{-7pt}

%
% Professional Experience
%
\Section{Experience}
\vspace{-4pt}
\ExperienceEntry{\textbf{Swift Platform Experience - Compiler Intern}}{May, 2023 - August, 2023}{Apple}{Manager - Richard Wei}{
    \vspace{-10pt}
    \begin{tightitemize}
        \item Swift compiler
            \subitem - Designed a new Intermediate Representation used internally across multiple teams at Apple.
            \subitem - Extended Swift's code generation and runtime to work with this new Intermediate Representation.
        \end{tightitemize}
}\vspace{4pt}\\
\ExperienceEntry{\textbf{Research Assisant}}{August, 2021 - February, 2023}{Purdue University}{Advisor - Tiark Rompf}{
    \vspace{-10pt}
    \begin{tightitemize}
        \item \href{https://dl.acm.org/doi/10.5555/3291168.3291227}{Flare}: In-memory query compiler backend for Apache Spark using generative programming
            \subitem - Developed a specialized file loader that reduces HDFS overheads by generating file topology aware native code.
            \subitem - Embedded a MapReduce-like framework in Flare to operate over files in HDFS.
        \end{tightitemize}
}\vspace{4pt}\\
\ExperienceEntry{\textbf{Google Summer of Code - Student}}{June, 2021 - August, 2021}{The LLVM Compiler Infrastructure Organization}{Supervisors - William Moses, Johannes Doerfert}{
    \vspace{-10pt} \begin{tightitemize}
    \item \href{https://enzyme.mit.edu/}{Enzyme}: LLVM Pass to perform automatic differentiation of statically analyzable LLVM IR
        \subitem - Integrated custom derivatives of several BLAS functions into Enzyme.
        \subitem - Wrote an LLVM pass to inline function definitions from bitcode files into LLVM IR.
    \end{tightitemize}
}\vspace{4pt}\\
\ExperienceEntry{\textbf{{\href{https://iris-hep.org/}{IRIS-HEP}} - Fellow}}{June, 2020 - September, 2020}{Princeton University}{Supervisor - Jim Pivarski}{
	\vspace{-10pt}
	\begin{tightitemize}
    \item \href{https://github.com/scikit-hep/awkward-1.0}{Awkward Array}: Library for nested, variable-sized data using NumPy-like idioms
        \subitem - Created a source to source compiler to generate equivalent Python for a subset of C++.
        \subitem - Created a property based testing framework.
        \subitem - Created a source to source compiler to generate equivalent parallel CUDA from specification (Python and type info).
	\end{tightitemize}
} \vspace{4pt}\\
\ExperienceEntry{\textbf{{\href{https://iris-hep.org/}{IRIS-HEP}} and \href{https://diana-hep.org/}{DIANA-HEP} - Fellow}}{June, 2018 - September, 2018; June, 2019 - September, 2019}{Fermi National Accelerator Laboratory and Princeton University}{Supervisor - Jim Pivarski}{
	\vspace{-10pt}
	\begin{tightitemize}
    \item \href{https://github.com/scikit-hep/uproot3.git}{Uproot}: Python implementation of ROOT I/O, an open source file format storing over an exabyte of HEP data
        \subitem - Enabled writing fundamental HEP data structures like TTrees and histograms to ROOT files.
        \subitem - Uproot has become one of the most widely used HEP libraries.
	\end{tightitemize}
}\\
\vspace{-20pt}


\Section{Other Open Source Contributions}
\vspace{-4pt}
\ExperienceEntry{\textit{Supervisors} - Nikos Vasilakis, Konstantinos Kallas}{February, 2022 - July, 2022}{}{}{
    \vspace{-23pt}
    \begin{tightitemize}
    \item \textbf{\href{https://binpash.github.io/web/}{PaSh}}: A system for parallelizing POSIX shell scripts
        \subitem - Helped extend PaSh for distributed file systems (HDFS)
    \end{tightitemize}
}\vspace{-16pt}\\
\ExperienceEntry{\textit{Supervisor} - Vassil Vassilev}{November, 2019 - May, 2021}{}{}{
    \vspace{-23pt}
    \begin{tightitemize}
    \item \textbf{\href{https://github.com/root-project/root}{ROOT}}: An open-source data analysis framework storing over an exabyte of data
        \subitem - Improvements to interpreter (rootcling)
    \item \textbf{\href{https://github.com/root-project/cling}{Cling}}: Interactive C++ interpreter built on top of Clang
        \subitem - Maintained cpt.py installer and packager
    \item \textbf{\href{https://clang.llvm.org/}{Clang}}: C language family frontend for LLVM
        \subitem - Several patches to print type information of C++ template arguments
    \end{tightitemize}
}\vspace{4pt}\\
\ExperienceEntry{\textit{Supervisor} - Jim Pivarski}{January, 2021 - February, 2021}{}{}{
    \vspace{-23pt}
    \begin{tightitemize}
    \item \textbf{\href{https://github.com/scikit-hep/awkward-1.0}{Awkward Array}} - Library for nested, variable-sized data using NumPy-like idioms
        \subitem - Created a parser for Awkward Array's type system
    \end{tightitemize}
} \vspace{-16pt}\\
\ExperienceEntry{\textit{Supervisors} - Jim Pivarski, Viktor Khristenko}{June, 2017 - August, 2017}{}{}{
    \vspace{-23pt}
    \begin{tightitemize}
    \item \textbf{\href{https://github.com/diana-hep/spark-root}{spark-root}} - Apache Spark datasource for ROOT
        	\subitem - Separated spark bindings from TTree reading code
        \item \textbf{\href{https://github.com/diana-hep/spark-root}{root4j}} - Java implementation of ROOT file reader
        	\subitem - Optimized codebase to facilitate interoperability
    \end{tightitemize}
}
\vspace{-3pt}

\Section{Teaching Experience}
\vspace{-4pt}
\textbf{CS 354: Operating Systems} - Purdue University \hfill{Fall 2022, Spring 2023, Fall 2023, Spring 2024}

\textbf{CS 240: Programming in C} - Purdue University \hfill{Fall 2021, Spring 2022}
\vspace{-6pt}

%
% Programming Skills
%
\Section{Programming Languages and Tools}
\vspace{-4pt}
\textbf{Experienced:} Python, C, CUDA

\textbf{Familiar:} C++, Java, Scala, Coq, ROOT, Bash, \LaTeX, Swift, WebAssembly
\vspace{-6pt}

%
% Workshops and Summer Schools
%
\Section{Summer Schools}
\vspace{-4pt}
\indent \EducationEntry{\textbf{{\href{https://www.cs.uoregon.edu/research/summerschool/summer21/index.php}{Oregon Programming Languages Summer School}}} - University of Oregon}{2021}{}{}\vspace{-15pt}\\
\indent \EducationEntry{\textbf{{\href{https://codas-hep.org/}{Computational and Data Science for High Energy Physics}}} - Princeton University}{2019}{}{}\\

\vspace{-34pt}

%
% Publications
%
\Section{Publications}
\small{\begin{tightitemize}
    \item T.Mustafa, K.Kallas, \textbf{P.Das}, N.Vasilakis, ``DiSh: Dynamic Shell-Script Distribution'', 20th USENIX Symposium on Networked Systems Design and Implementation (NSDI 2023).
    \item J.Pivarski, I.Osborne, {\textbf{P.Das}}, D.Lange, P.Elmer, {\href{https://www.epj-conferences.org/articles/epjconf/abs/2021/05/epjconf_chep2021_03002/epjconf_chep2021_03002.html}{``AwkwardForth: accelerating Uproot with an internal DSL''}}, 25th International Conference on Computing in High-Energy and Nuclear Physics (vCHEP, 2021), DOI: 10.1051/epjconf/202125103002.
    \item J.Pivarski, I.Osborne, {\textbf{P.Das}}, A.Biswas, P.Elmer, {\href{http://conference.scipy.org/proceedings/scipy2020/jim_pivarski.html}{``Awkward Array: JSON-like data, NumPy-like idioms''}}, Proceedings of the 19th Python in Science Conference (SciPy USA, 2020), Pages 68-74, DOI: 10.25080/Majora-342d178e-00b.
    \item E.Rodrigues, et al., {\href{https://www.epj-conferences.org/articles/epjconf/abs/2020/21/epjconf_chep2020_06028/epjconf_chep2020_06028.html}{``The Scikit HEP Project - overview and prospects''}}, Proceedings of the 24th International Conference on Computing in High Energy and Nuclear Physics (CHEP 2019), DOI: 10.1051/epjconf/202024506028.
    \item N.Saha, {\textbf{P.Das}}, H.N.Saha, {\href{https://www.inderscienceonline.com/doi/abs/10.1504/IJAPR.2018.094819}{``Authorship Attribution of Short Texts using a Multi Layer Perceptron''}}, International Journal of Applied Pattern Recognition, 2018 Vol. 5 No. 3, Pages 251-259, DOI: 10.1504/IJAPR.2018.10016100.
\end{tightitemize}}

%
% Presentations
%
\Section{Invited talks at Conferences}
\begin{tightitemize}
\item \href{https://youtu.be/mxI9fYbpndI}{GSoC Experience - Enzyme} (\href{https://llvm.swoogo.com/2021devmtg/}{LLVM Developers' Meeting}) \hfill{2021}
\item \href{https://youtu.be/jClVsR6XfdI}{Python in High Energy Physics} (\href{https://scipy.in/2019}{SciPy India}, \href{https://us.pycon.org/2020/}{PyCon USA}) \hfill{2019, 2020}
\item \href{https://indico.cern.ch/event/833895/contributions/3577892/attachments/1927752/3191883/uproot-pyhep.pdf}{Writing files with uproot} (\href{https://indico.cern.ch/event/833895/}{PyHEP}) \hfill{2019}
\item \href{https://indico.cern.ch/event/697389/contributions/3102807/attachments/1713054/2762448/Writing_files_with_uproot.pdf}{Writing files with uproot} (\href{https://indico.cern.ch/event/697389/}{ROOT Users' Workshop}) \hfill{2018}
\end{tightitemize}

\Section{Invited talks at External Research Group Meetings}
\begin{tightitemize}
\item \href{https://indico.cern.ch/event/946427/contributions/3976986/attachments/2094014/3519161/IRIS-HEP-Fellow-Awkward.pdf}{Language Transformations for the Awkward Array library} (\href{https://indico.cern.ch/event/946427/}{IRIS-HEP Fellow Presentations}) \hfill{2020}
\item CUDA backend for the Awkward Array project (\href{https://liberty.princeton.edu/}{Princeton University Liberty Research Group}) \hfill{2020}
\item \href{https://indico.cern.ch/event/840667/contributions/3527109/attachments/1908764/3153297/uproot-irisfellow-final.pdf}{Writing TTrees with uproot} (\href{https://indico.cern.ch/event/840667/}{IRIS-HEP Topical Meeting: Summer student project presentations}) \hfill{2019}
\item \href{https://indico.cern.ch/event/754335/contributions/3166239/attachments/1734208/2804184/Writing_files_with_uproot_-_DIANA_HEP.pdf}{Writing files with uproot} (\href{https://indico.cern.ch/event/754335/}{DIANA Meeting: Updates on ROOT I/O}) \hfill{2018}
\item \href{https://indico.cern.ch/event/658754/contributions/2685907/attachments/1506368/2347492/Refactoring_code_from_spark-root_to_root4j.pdf}{Separation of Concerns - Refactoring code between ROOT4J and Spark-Root} (\href{https://indico.cern.ch/event/658754/}{CMS Big Data Science}, \href{https://indico.cern.ch/event/655833/}{DIANA-HEP}) \hfill{2017}
\end{tightitemize}

\Section{Service}
\begin{tightitemize}
\item Artifact Evaluation Committee - SOSP 2023
\end{tightitemize}

\end{document}
