\documentclass[10pt, letterpaper]{fulldeps}
%
% Preamble
%
\usepackage{graphicx}

\pagenumbering{gobble}
\setlength{\columnseprule}{0.2pt}

%
% Document Environment
%
\begin{document}

%
% Name
%
\vspace*{-0.8cm}
\textbf{\fontsize{25}{25}\selectfont Pratyush Das}
\vspace{0.4cm}

%
% Personal Entries
%
\vspace{-45pt}
\begin{spacing}{1}
\PersonalEntry{\textbullet{} Phone}{(+91) 9051603323}
\PersonalEntry{\textbullet{} Email}{\href{mailto:reikdas@gmail.com}{reikdas@gmail.com}}
\PersonalEntry{\textbullet{} GitHub}{\urlstyle{same}\url{https://github.com/reikdas}}
\end{spacing}
\vspace{-13pt}

%
% Education
%
\vspace{10pt}
\Section{Education}
\vspace{-6pt}
\EducationEntry{\textbf{Institute of Engineering \& Management, Kolkata}}{2017-2021(Expected)}{Bachelor of Technology in Computer Science and Engineering. CGPA: 8.00/10}{}\\
\EducationEntry{\textbf{Don Bosco School, Park Circus}}{2016}{High School}{}
\vspace{-10pt}

%
% Professional Experience
%
\Section{Experience}
\vspace{-4pt}
\ExperienceEntry{\textbf{{\href{https://iris-hep.org/}{IRIS-HEP}} - Fellow}}{June, 2020 - August, 2019}{}{Supervisor - Dr. Jim Pivarski(Princeton University)}{
	\vspace{-10pt}
	\begin{tightitemize}
    \item Awkward Array: Library for nested, variable-sized data using NumPy-like idioms
        \subitem - Designed a source to source compiler to generate equivalent Python for a subset of C++.
	\end{tightitemize}
} \vspace{-3pt}\\
\ExperienceEntry{\textbf{{\href{https://iris-hep.org/}{IRIS-HEP}} - Fellow}}{June, 2019 - September, 2019}{Fermi National Accelerator Laboratory, USA - LHC Physics Centre}{Supervisor - Dr. Jim Pivarski(Princeton University)}{
	\vspace{-10pt}
	\begin{tightitemize}
    \item uproot: Python implementation of ROOT, the open source file format storing the largest quantity of data in the world
            \subitem - Added functionality to write ROOT files with TTrees.
            \subitem - Played a major role in making uproot one of the most widely used High Energy Physics libraries.
	\end{tightitemize}
} \vspace{3pt}\\
\ExperienceEntry{\textbf{{\href{http://diana-hep.org/}{DIANA-HEP}} - Fellow}}{June, 2018 - September, 2018}{Fermi National Accelerator Laboratory, USA - LHC Physics Centre}{Supervisor - Dr. Jim Pivarski(Princeton University)}{
      \vspace{-10pt}
      \begin{tightitemize}
          \item uproot
            \subitem - Examined ROOT serialization of objects.
            \subitem - Added functionality to write ROOT files with strings and histograms.
      \end{tightitemize}
} \vspace{3pt}\\
\ExperienceEntry{\textbf{{\href{http://diana-hep.org/}{DIANA-HEP}} - Summer Student}}{June, 2017 - August, 2017}{}{Supervisors - Dr. Jim Pivarski(Princeton University), Dr. Viktor Khristenko(CERN)}{
    \vspace{-10pt}
    \begin{tightitemize}
    	\item spark-root - Apache Spark datasource for ROOT
        	\subitem - Separated spark bindings from TTree reading code.
        \item root4j - Java implementation of ROOT file reader
        	\subitem - Optimized codebase to facilitate interoperability
    \end{tightitemize}
}
\vspace{-3pt}

%
% Workshops and Summer Schools
%
\Section{Summer Schools}
\vspace{-5pt}
\EducationEntry{\textbf{{\href{https://codas-hep.org/}{Computational and Data Science for High Energy Physics}}}}{2019}{Princeton University}{}\\
\vspace{-20pt}

%
% Programming Skills
%
\Section{Programming Skills}
\vspace{-5pt}
\textbf{Languages:} Python, Java, C, C++

\textbf{Libraries/Frameworks:} numpy, ROOT, git, CUDA, *nix
\vspace{-6pt}

%
% Publications
%
\Section{Publications}
\small{\begin{tightitemize}
    \item J.Pivarski, I.Osborne, {\textbf{P.Das}}, A.Biswas, P.Elmer, {\href{http://conference.scipy.org/proceedings/scipy2020/jim_pivarski.html}{``Awkward Array: JSON-like data, NumPy-like idioms''}}, Proceedings of the 19th Python in Science Conference. \hfill{2020}
    \item N.Saha, {\textbf{P.Das}}, H.N.Saha, {\href{https://www.inderscienceonline.com/doi/abs/10.1504/IJAPR.2018.094819}{``Authorship Attribution of Short Texts using a Multi Layer Perceptron''}}, International Journal of Applied Pattern Recognition, 2018 Vol. 5 No. 3, Pages 251-259, DOI: 10.1504/IJAPR.2018.10016100. \hfill{2018}
\end{tightitemize}}

%
% Presentations
%
\Section{Conference Talks}
\vspace{-5pt}
\PresentationEntry{\textbullet{} {\href{https://youtu.be/jClVsR6XfdI}{Python in High Energy Physics.}}}{\small{-{\href{https://us.pycon.org/2020/}{PyCon USA}} (Remote)}} \hfill{2020}\\
\PresentationEntry{\textbullet{} {\href{https://static.fossee.in/scipy2019/SciPyTalks/SciPyIndia2019_S008_Python_in_High_Energy_Physics_20191130.mp4}{Python in High Energy Physics}}}{\small{-{\href{https://scipy.in/2019}{Scipy India}} (Indian Institute of Technology, Bombay)}} \hfill{2019}\\
\PresentationEntry{\textbullet{} {\href{https://indico.cern.ch/event/773049/contributions/3476182/attachments/1938227/3213768/EduardoRodrigues_2019-11-05_CHEP2019Adelaide.pdf}{The Scikit-HEP Project: Overview and Prospects} - \underline{Eduardo Rodrigues} et al.}}{\small{-{\href{http://chep2019.org/}{24th International Conference on Computing in High Energy and Nuclear Physics}} (University of Adelaide)}} \hfill{2019}\\
\PresentationEntry{\textbullet{} {\href{https://indico.cern.ch/event/833895/contributions/3577892/attachments/1927752/3191883/uproot-pyhep.pdf}{Writing files with uproot}}}{\small{-{\href{https://indico.cern.ch/event/833895/}{PyHEP}} (Abington, UK)}} \hfill{2019}\\
\PresentationEntry{\textbullet{} {\href{https://indico.cern.ch/event/697389/contributions/3102807/attachments/1713054/2762448/Writing_files_with_uproot.pdf}{Writing files with uproot}}}{\small{-{\href{https://indico.cern.ch/event/697389/}{ROOT Users' Workshop}} (Academy of Sciences and Arts of Bosnia and Herzegovina)}} \hfill{2018}\\
\vspace{-15pt}

\Section{Talks at Meetings}
\vspace{-5pt}
\PresentationEntry{\textbullet{} {\href{https://indico.cern.ch/event/913583/contributions/3841826/attachments/2027488/3392299/PR5297_--_Testing_Facilities.pdf}{PR 5297: Testing Facilities} - \underline{Vassil Vassilev}, Pratyush Das}}{\small{-{\href{https://indico.cern.ch/event/913583/}{ROOT Team Meeting}}(Vidyo)}} \hfill{2020}\\
\PresentationEntry{\textbullet{} {\href{https://indico.cern.ch/event/840667/contributions/3527109/attachments/1908764/3153297/uproot-irisfellow-final.pdf}{Writing TTrees with uproot}}}{\small{-{\href{https://indico.cern.ch/event/840667/}{IRIS-HEP Topical Meeting: Summer student project presentations}}(Vidyo)}} \hfill{2019}\\
\PresentationEntry{\textbullet{} {\href{https://indico.cern.ch/event/697389/contributions/3102807/attachments/1713054/2762448/Writing_files_with_uproot.pdf}{Writing files with uproot}}}{\small{-{\href{https://indico.cern.ch/event/754335/}{DIANA Meeting: Updates on ROOT I/O}}(Vidyo)}} \hfill{2018}\\
\PresentationEntry{\textbullet{} {\href{https://indico.cern.ch/event/658754/contributions/2685907/attachments/1506368/2347492/Refactoring_code_from_spark-root_to_root4j.pdf}{Separation of Concerns - Refactoring code between ROOT4J and Spark-Root}}}{\small{-{\href{https://indico.cern.ch/event/655833/}{DIANA Meeting: Student Projects}}(Vidyo); {\href{https://indico.cern.ch/event/658754/}{CMS Big Data Science Projects}}(Vidyo)}}\hfill{2017}\\
\vspace{-15pt}

%
% Patents
%
%\Section{Patents}
%\small{\begin{tightitemize}
%    \item Autonomous navigation system for driverless cars, Application no. - 201831024950. {\textit{Patent pending}}. \hfill{2018}
%\end{tightitemize}}

%
% Achievements
%
\Section{Academic Achievements}
\begin{tightitemize}
\item Awarded the IRIS-HEP undergraduate fellowship. \hfill{2020}
    \item Awarded travel grant to speak at PyCon USA 2020 in Pittsburgh, USA. \hfill{2020}
    \item Awarded travel grant to attend PLMW and POPL 2020 in New Orleans, USA. \hfill{2019}
    \item Awarded travel grant to attend CoDaS-HEP summer school at Princeton University. \hfill{2019}
    \item Awarded the IRIS-HEP undergraduate fellowship. \hfill{2019}
    \item Awarded travel grant to speak at ROOT Users' Workshop 2018 in Sarajevo, Bosnia and Herzegovina. \hfill{2018}
	\item Awarded the DIANA-HEP undergaduate felowship. \hfill{2018}
\end{tightitemize}

\Section{Extracurricular Achievements}
\begin{tightitemize}
    \item International Rated Chess Player (Federation Internationale des Echecs) \hfill{2016}
    \item Adhyayan National Student Leadership Contest (Adhyayan India) - Third \hfill{2015}
    \item IT Quiz (Computer Society of India) - Second \hfill{2014}
\end{tightitemize}

\Section{Open Source Projects}
\begin{tightitemize}
\item {\href{https://github.com/scikit-hep/uproot}{uproot}} (Core developer) - Designed ROOT file writing interface.
\item {\href{https://github.com/scikit-hep/awkward-1.0}{Awkward Array}} - Designed transpilers from a subset of C++ to Python and parallelized CUDA.
\item {\href{https://github.com/scikit-hep/uproot-methods}{uproot-methods}} - Enabled support to recognize hook for multidimensional uproot histograms.
\item {\href{https://github.com/diana-hep/root4j}{root4j}} - Optimized interface for interoperability.
\item {\href{https://github.com/diana-hep/spark-root}{spark-root}} - Separated spark bindings from TTree reading code.
\item {\href{https://github.com/root-project/cling}{cling}} - Configured installer to build using LLVM binary.
\item {\href{https://github.com/root-project/root}{ROOT}} - Provided fixes to rootcling bugs.
\end{tightitemize}

\Section{Featured in Media}
\begin{tightitemize}
\item \textit{\href{https://www.princeton.edu/news/2019/08/19/princeton-leads-efforts-develop-national-data-training-framework-high-energy}{Princeton leads efforts to develop national data training framework for high energy physics}} - Princeton University News \hfill{2019}
\end{tightitemize}

\end{document}
