\documentclass[10pt, letterpaper]{fulldeps}
%
% Preamble
%
\usepackage{graphicx}

\pagenumbering{gobble}
\setlength{\columnseprule}{0.2pt}

%
% Document Environment
%
\begin{document}

%
% Name
%
\vspace*{-0.8cm}
\textbf{\fontsize{25}{25}\selectfont Pratyush Das}
\vspace{0.4cm}

%
% Personal Entries
%
\vspace{-45pt}
\begin{spacing}{1}
\PersonalEntry{\textbullet{} Phone}{(+91) 9051603323}
\PersonalEntry{\textbullet{} Email}{\href{mailto:reikdas@gmail.com}{reikdas@gmail.com}}
\PersonalEntry{\textbullet{} GitHub}{\urlstyle{same}\url{https://github.com/reikdas}}
\end{spacing}
\vspace{-13pt}

%
% Education
%
\vspace{10pt}
\Section{Education}
\vspace{-6pt}
\EducationEntry{\textbf{Institute of Engineering \& Management, Kolkata}}{2017-2021(Expected)}{Bachelor of Technology in Computer Science and Engineering. CGPA: 8.04/10}{}\\
\vspace{-20pt}

%
% Professional Experience
%
\Section{Experience}
\vspace{-4pt}
\ExperienceEntry{\textbf{{\href{https://iris-hep.org/}{IRIS-HEP}} - Fellow}}{June, 2020 - September, 2020}{Awarded the IRIS-HEP Fellowship by Princeton University}{\textit{Supervisor:} Dr. Jim Pivarski(Princeton University)}{
	\vspace{-10pt}
	\begin{tightitemize}
    \item {\href{https://github.com/scikit-hep/awkward-1.0}{Awkward Array}}: Library for nested, variable-sized data using NumPy-like idioms
        \subitem - Created source to source compilers to generate equivalent Python and parallel CUDA from a subset of C++.
        \subitem - Created a property based testing framework.
	\end{tightitemize}
} \vspace{3pt}\\
\ExperienceEntry{\textbf{{\href{https://iris-hep.org/}{IRIS-HEP}} - Fellow}}{June, 2019 - September, 2019}{Awarded the IRIS-HEP Fellowship by Princeton University}{\textit{Supervisor:} Dr. Jim Pivarski(Princeton University); \textit{Location:} Fermi National Accelerator Laboratory, USA}{
	\vspace{-10pt}
	\begin{tightitemize}
    \item {\href{https://github.com/scikit-hep/uproot}{uproot}}: Python implementation of ROOT I/O, an open source file format storing over an exabyte of HEP data
        \subitem - Completed ROOT file writing interface by adding functionality to write ROOT files with TTrees.
        \subitem - uproot has become one of the most widely used High Energy Physics libraries (100K+ downloads).
	\end{tightitemize}
} \vspace{3pt}\\
\ExperienceEntry{\textbf{{\href{http://diana-hep.org/}{DIANA-HEP}} - Fellow}}{June, 2018 - September, 2018}{Awarded the DIANA-HEP Undergraduate Fellowship by Princeton University}{\textit{Supervisor:} Dr. Jim Pivarski(Princeton University); \textit{Location:} Fermi National Accelerator Laboratory, USA}{
      \vspace{-10pt}
      \begin{tightitemize}
          \item {\href{https://github.com/scikit-hep/uproot}{uproot}}:
              \subitem - Co-developed the uproot library with Jim Pivarski; authored the first ever ROOT file writing interface in Python.
      \end{tightitemize}
} \vspace{-3pt}\\
\ExperienceEntry{\textbf{{\href{http://diana-hep.org/}{DIANA-HEP}} - Summer Student}}{June, 2017 - August, 2017}{}{\textit{Supervisors:} Dr. Jim Pivarski(Princeton University), Dr. Viktor Khristenko(CERN)}{
    \vspace{-10pt}
    \begin{tightitemize}
    \item {\href{https://github.com/diana-hep/spark-root}{spark-root}} and {\href{https://github.com/diana-hep/root4j}{root4j}}: Set of libraries to read ROOT files into Apache Spark dataframes
    	    \subitem - Refactored Apache Spark bindings from ROOT TTree reading code in spark-root for interoperability with root4j.
    \end{tightitemize}
}
\vspace{-10pt}

%
% Workshops and Summer Schools
%
\Section{Summer Schools}
\vspace{-5pt}
\EducationEntry{\textbf{{\href{https://codas-hep.org/}{Computational and Data Science for High Energy Physics}}}}{2019}{Princeton University}{}\\
\vspace{-15pt}
\begin{tightitemize}
\item Interviewed - \textit{\href{https://www.princeton.edu/news/2019/08/19/princeton-leads-efforts-develop-national-data-training-framework-high-energy}{Princeton University News}} \hfill{2019}
\end{tightitemize}
\vspace{-5pt}

%
% Programming Skills
%
\Section{Programming Languages and Tools}
\vspace{-3pt}
\textbullet{}Python \textbullet{}C \textbullet{}C++ \textbullet{}Java \textbullet{}CUDA \textbullet{}*nix \textbullet{}ROOT
\vspace{-4pt}

%
% Publications
%
\Section{Publications}
\small{\begin{tightitemize}
    \item J.Pivarski, I.Osborne, {\textbf{P.Das}}, A.Biswas, P.Elmer, {\href{http://conference.scipy.org/proceedings/scipy2020/jim_pivarski.html}{``Awkward Array: JSON-like data, NumPy-like idioms''}}, Proceedings of the 19th Python in Science Conference (SciPy, USA), 2020, Pages 68-74, DOI: 10.25080/Majora-342d178e-00b. \hfill{2020}
    \item E.Rodrigues, et al., {\href{https://arxiv.org/abs/2007.03577}{``The Scikit HEP Project - overview and prospects``}}, Proceedings of the 24th International Conference on Computing in High Energy and Nuclear Physics (CHEP 2019), Adelaide, Australia, 2019. \hfill{2020}
    \item N.Saha, {\textbf{P.Das}}, H.N.Saha, {\href{https://www.inderscienceonline.com/doi/abs/10.1504/IJAPR.2018.094819}{``Authorship Attribution of Short Texts using a Multi Layer Perceptron''}}, International Journal of Applied Pattern Recognition, 2018 Vol. 5 No. 3, Pages 251-259, DOI: 10.1504/IJAPR.2018.10016100. \hfill{2018}
\end{tightitemize}}

%
% Presentations
%
\Section{Select Talks}
\vspace{-5pt}
\PresentationEntry{\textbullet{} Language Transformations for the Awkward Array library}{\small{-{\href{https://indico.cern.ch/event/946427/contributions/3976986/attachments/2094014/3519161/IRIS-HEP-Fellow-Awkward.pdf}{IRIS-HEP Fellow Presentations}} (Remote)}} \hfill{2020}\\
\PresentationEntry{\textbullet{} CUDA backend for the Awkward Array project}{-Princeton University Liberty Research Group Meeting (Remote)} \hfill{2020}\\
\PresentationEntry{\textbullet{} Python in High Energy Physics}{\small{-{\href{https://us.pycon.org/2020/speaker/profile/217/}{PyCon USA}} (Remote)}} \hfill{2020}\\
\hspace*{1ex}\hspace*{1ex}\hspace*{1ex}\hspace*{1ex}\hspace*{1ex}\hspace*{1ex}\hspace*{1ex}\hspace*{0.4ex}\textit{{\href{https://static.fossee.in/scipy2019/slides/python_in_high_enrgy_phy_pratyush_das.pdf}{-SciPy India}} (Indian Institute of Technology, Bombay)}\hfill{2019}\\
\PresentationEntry{\textbullet{} The Scikit-HEP Project: Overview and Prospects - \underline{Eduardo Rodrigues} et al.}{\small{-{\href{https://indico.cern.ch/event/773049/contributions/3476182/}{24th International Conference on Computing in High Energy and Nuclear Physics}} (University of Adelaide)}} \hfill{2019}\\
\PresentationEntry{\textbullet{} Writing TTrees with uproot}{\small{-{\href{https://indico.cern.ch/event/840667/contributions/3527109/attachments/1908764/3153297/uproot-irisfellow-final.pdf}{IRIS-HEP Topical Meeting: Summer student project presentations}} (Remote)}} \hfill{2019}\\
\PresentationEntry{\textbullet{} Writing files with uproot}{\small{-{\href{https://indico.cern.ch/event/833895/contributions/3577892/}{PyHEP}} (Abington, UK)}} \hfill{2019}\\
\hspace*{1ex}\hspace*{1ex}\hspace*{1ex}\hspace*{1ex}\hspace*{1ex}\hspace*{1ex}\hspace*{1ex}\hspace*{0.4ex}\textit{{\href{https://indico.cern.ch/event/697389/contributions/3102807/}{-ROOT Users' Workshop}} (Academy of Sciences and Arts of Bosnia and Herzegovina)}\hfill{2018}\\
\PresentationEntry{\textbullet{} Separation of Concerns - Refactoring code between ROOT4J and Spark-Root}{\small{-{\href{https://indico.cern.ch/event/658754/contributions/2685907/attachments/1506368/2347492/Refactoring_code_from_spark-root_to_root4j.pdf}{CMS Big Data Science Projects}} (Remote)}}\hfill{2017}\\
\vspace{-15pt}

%
% Patents
%
%\Section{Patents}
%\small{\begin{tightitemize}
%    \item Autonomous navigation system for driverless cars, Application no. - 201831024950. {\textit{Patent pending}}. \hfill{2018}
%\end{tightitemize}}

%
% Achievements
%
\Section{Other Achievements}
\begin{tightitemize}
    \item Awarded travel grant and selected to attend PLMW and POPL 2020 in New Orleans, USA. \hfill{2019}
    \item International Rated Chess Player (Federation Internationale des Echecs) \hfill{2016}
    \item Adhyayan National Student Leadership Contest (Adhyayan India) - Third \hfill{2015}
    \item IT Quiz (Computer Society of India) - Second \hfill{2014}
\end{tightitemize}

\Section{Other Major Open Source Contributions}
\begin{tightitemize}
\item {\href{https://github.com/root-project/cling}{cling}} - Configured installer to build using LLVM binary.[Supervised by Dr. Vassil Vassilev(Princeton University)]
\item {\href{https://github.com/root-project/root}{ROOT}} - Added ROOTUnitTestSupport and fixed several rootcling bugs.[Supervised by Dr. Vassil Vassilev(Princeton University)]
\item {Clang} - Upstreaming patches from Cling.[Supervised by Dr. Vassil Vassilev(Princeton University)]
\end{tightitemize}

\end{document}
